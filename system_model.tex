\chapter{System Model}
\label{cha:system}
\vspace{0.4 cm}

In this chapter, the model of the proposed system is presented.
In the first section, an in-depth presentation of the designed system architecture is provided.
The different components of the system with their functionalities and interactions are described.
Lastly, the modules focused on the three use cases of interest are treated more in detail in dedicated sections.
After this chapter, it will be clear what the main parts of the system are and how they cooperate to accomplish the various use cases of interest.


\section{System architecture}
\label{sec:architecture}
\vspace{0.2 cm}

The use case diagram for the proposed system is presented in figure~\ref{fig:usecase}.
Users can interact with the system via dedicated APIs.
To be authorized, a user must be authenticated and have an account with an active subscription.
It is also possible to have more users for the same account.
It is possible to request the following asynchronous operations:
\begin{itemize}
  \item Load new data with a create or update logic specifying the type of data. It is supported the upload of CSV files;
  \item Train new models based on available data for a specific use case specifying the time granularity (hourly or daily). The supported specific use cases for the models that will be used in the dedicated forecast requests are electricity demand forecasting, consumption baseline forecasting, and electricity production forecasting;
  \item Forecast future data for a specific use case using a specified model for a certain time granularity, for a certain starting date and time horizon.
\end{itemize}

In addition, a task scheduler periodically triggers the performance evaluation of the available models and if needed triggers the models re-training for having up-to-date models with respect to the user data so that they might perform better on future forecasts.

\begin{figure}[H]
\centering
\includegraphics[width=0.6\textwidth]{images/architecture_use_case}
\caption{The use case diagram for the proposed system.}
\label{fig:usecase}
\end{figure}

For achieving the presented use cases the system architecture shown in figure~\ref{fig:components} was designed.
It is composed of the following components:
\begin{itemize}
  \item Authentication layer: manages the authentication process and permissions of users;
  \item API layer: manages the user interactions with the system;
  \item Task manager: coordinates the execution of the tasks;
  \item Data loading task: parses and stores the users' loaded data;
  \item ML model interface: handles the interactions with the ML models storage where the trained ML models are stored;
  \item Model training task: trains new models based on available data;
  \item Forecasting task: forecasts future data using a specified model;
  \item Task scheduler: manages the periodic scheduled tasks;
  \item Performance evaluation task: evaluates the performance of the available models and if needed triggers the models re-training.
\end{itemize}

The system interacts with the following external components:
\begin{itemize}
  \item Users DB: used to store and retrieve user data, including account data;
  \item Data DB: used to store and retrieve the users' loaded data;
  \item Weather Data DB: used to retrieve the weather data;
  \item ML models storage: used to store and retrieve the trained models;
  \item Forecasts DB: used to store and retrieve the computed forecasts.
\end{itemize}

\begin{figure}[H]
\centering
\includegraphics[width=0.9\textwidth]{images/architecture_components}
\caption{The architecture designed for the proposed system.}
\label{fig:components}
\end{figure}

Figure~\ref{fig:interactions} shows the interactions among the components of the designed architecture for achieving the presented use cases.
In the following subsections, the logic for each use case is explained with the relative components' interactions.

\begin{figure}[H]
\centering
\includegraphics[width=1\textwidth]{images/architecture_interactions}
\caption{The interactions among the components in the designed architecture.}
\label{fig:interactions}
\end{figure}


\vspace{0.1 cm}
\subsection{Data loading}
\label{sec:loading}
\vspace{0.1 cm}

The diagram representing the components' interactions for achieving the data loading use case is presented in figure~\ref{fig:loadinginteractions}.

The user sends the data to the system.
First, the authentication layer verifies the authentication of the user and checks whether it has a subscription to load the data.
The endpoint dedicated to data loading takes charge of the user request and provides the task manager with the details for creating a data loading task.
The identifier of the task is then returned to the user, who can request the status of the data loading request and check the result of the operation when completed.

\begin{figure}[H]
\centering
\includegraphics[width=1\textwidth]{images/architecture_data_loading_interactions}
\caption{The interactions among the components for achieving the data loading use case.}
\label{fig:loadinginteractions}
\end{figure}

The data loading task starts when the task manager schedules it.
As shown by the diagram in figure~\ref{fig:loadingflow}, the logic inside this task can be formulated in 2 steps: parse the data and store the data.
The data parser block transforms the user data in a platform-compatible format and passes it to the data storage block which stores the data inside the data DB.
If there is no model for the specific use case for which data is loaded, the task manager will also create a model training task to train all the available models on this data.
This allows the system to be ready for the forecast requests on the loaded data.
The model training task is described more in detail in subsection\ref{sec:training}.

\begin{figure}[H]
\centering
\includegraphics[width=1\textwidth]{images/architecture_data_loading_flow}
\caption{Diagram representing the data loading flow.}
\label{fig:loadingflow}
\end{figure}

The sequence diagram representing the complete data loading procedure is presented in figure~\ref{fig:loadingsequence}.

\begin{figure}[H]
\centering
\includegraphics[width=1\textwidth]{images/architecture_data_loading_sequence}
\caption{Sequence diagram of the data loading procedure.}
\label{fig:loadingsequence}
\end{figure}


\vspace{0.1 cm}
\subsection{Model training}
\label{sec:training}
\vspace{0.1 cm}

The diagram representing the components' interactions for achieving the model training use case is presented in figure~\ref{fig:traininginteractions}.

The user sends the specification of the model to train to the system.
First, the authentication layer verifies the authentication of the user and checks whether it has a subscription to request the training of a model.
The endpoint dedicated to model training takes charge of the user request and provides the task manager with the details for creating a model training task.
The identifier of the task is then returned to the user, who can request the status of the model training request and get the trained model when completed.

\begin{figure}[H]
\centering
\includegraphics[width=1\textwidth]{images/architecture_training_interactions}
\caption{The interactions among the components for achieving the model training use case.}
\label{fig:traininginteractions}
\end{figure}

The model training task starts when the task manager schedules it.
As shown by the diagram in figure~\ref{fig:trainingflow}, the logic inside this task can be formulated in 2 steps: preprocess the data and train the model.
The data preprocessing block collects the data for the specific use case for which the model will be trained, aggregates it correctly, cleans it, and enriches it with weather data.
The models training module takes the preprocessed data and trains the model with the provided specifications.
The resulting model is then passed to the ML models interface which stores it inside the ML models storage.

\begin{figure}[H]
\centering
\includegraphics[width=0.7\textwidth]{images/architecture_training_flow}
\caption{Diagram representing the model training flow.}
\label{fig:trainingflow}
\end{figure}

The sequence diagram representing the complete model training procedure is presented in figure~\ref{fig:trainingsequence}.

\begin{figure}[H]
\centering
\includegraphics[width=0.8\textwidth]{images/architecture_training_sequence}
\caption{Sequence diagram representing the model training procedure.}
\label{fig:trainingsequence}
\end{figure}


\vspace{0.1 cm}
\subsection{Forecasting}
\label{sec:forecasting}
\vspace{0.1 cm}

The diagram representing the components' interactions for achieving the forecasting for a specific use case is presented in figure~\ref{fig:forecastinginteractions}.

\begin{figure}[H]
\centering
\includegraphics[width=1\textwidth]{images/architecture_forecasting_interactions}
\caption{The interactions among the components for achieving the forecasting for a specific use case.}
\label{fig:forecastinginteractions}
\end{figure}

The user sends the specification for computing forecasts for a specific use case to the system.
First, the authentication layer verifies the authentication of the user and checks whether it has a subscription to request the forecasts for a specific use case.
The endpoint dedicated to forecasting takes charge of the user request and provides the task manager with the details for creating a forecasting task.
The identifier of the task is then returned to the user, who can request the status of the forecasting request and get the computed forecasts when completed.

The forecasting task starts when the task manager schedules it.
As shown by the diagram in figure~\ref{fig:forecastflow}, the logic inside this task can be formulated in 2 steps: preprocess the data and forecast future data.
The data preprocessing block collects the needed data for the specific use case for which the forecasts will be made, aggregates it correctly, cleans it, and enriches it with weather data. In addition to past weather data also weather forecasts are collected to help produce more accurate future data.
The prediction module takes the preprocessed data, requests the specified model, and computes the forecasts.
The computed forecasts are then stored in the forecasts DB.

\begin{figure}[H]
\centering
\includegraphics[width=0.6\textwidth]{images/architecture_forecasting_flow}
\caption{Diagram representing the forecasting flow.}
\label{fig:forecastflow}
\end{figure}

The sequence diagram representing the complete forecasting procedure for a specific use case is presented in figure~\ref{fig:forecastingsequence}.

\begin{figure}[H]
\centering
\includegraphics[width=0.8\textwidth]{images/architecture_forecasting_sequence}
\caption{Sequence diagram representing the forecasting procedure.}
\label{fig:forecastingsequence}
\end{figure}


\vspace{0.1 cm}
\subsection{Performance evaluation}
\label{sec:scheduler}
\vspace{0.1 cm}

The diagram representing the components' interactions for achieving the performance evaluation use case is presented in figure~\ref{fig:schedulerinteractions}.

The task scheduler periodically creates and triggers a performance evaluation task.
As shown by the diagram in figure~\ref{fig:schedulerflow}, the logic inside the performance evaluation task can be formulated in 2 steps: preprocess the data and evaluate the performance.
The data preprocessing block collects the data for the specific use case for which the model will be trained, aggregates it correctly, cleans it, and enriches it with weather data.
The performance evaluation module takes the preprocessed data and evaluates the performance of the last available models.
When performance drops significantly, it triggers the models re-training by creating a model training task for having up-to-date models with respect to the user data so that they might perform better on future forecasts.
The model training task is described more in detail in subsection\ref{sec:training}.

\begin{figure}[H]
\centering
\includegraphics[width=1\textwidth]{images/architecture_scheduler_interactions}
\caption{The interactions among the components for achieving the performance evaluation use case.}
\label{fig:schedulerinteractions}
\end{figure}

\begin{figure}[H]
\centering
\includegraphics[width=1\textwidth]{images/architecture_scheduler_flow}
\caption{Diagram representing the performance evaluation flow.}
\label{fig:schedulerflow}
\end{figure}

The sequence diagram representing the complete performance evaluation procedure is presented in figure~\ref{fig:schedulersequence}.

\begin{figure}[H]
\centering
\includegraphics[width=0.6\textwidth]{images/architecture_scheduler_sequence}
\caption{Sequence diagram representing the performance evaluation procedure.}
\label{fig:schedulersequence}
\end{figure}


\section{System's common components}
\label{sec:components}
\vspace{0.2 cm}

Describe the system's common components between the various specific use cases:
\begin{itemize}
  \item Data parser: ... ;
  \item Data storage: ... ;
  \item Data preprocessing: (with also the data enrichment) ... ;
  \item Models training module: ... ;
  \item Prediction module: ... ;
  \item Performance evaluation module: ... .
\end{itemize}


\section{Electricity demand forecasting module}
\label{sec:demandmodel}
\vspace{0.2 cm}

Describe the electricity demand forecasting module ...

How the module works, what contains, describe the models used for the task-specific and then report also in the sections below if they are common, ...

The system model training schematic representation for electricity demand forecasting is presented in figure~\ref{fig:modeltrainingdemand}.

% TODO descrivere specializzazione della parte sopra

\begin{figure}[H]
\centering
\includegraphics[width=0.7\textwidth]{images/system_model_training_demand}
\caption{The system model training schematic representation for electricity demand forecasting.}
\label{fig:modeltrainingdemand}
\end{figure}

The system model forecasting schematic representation for electricity demand forecasting is presented in figure~\ref{fig:modelforecastingdemand}.

\begin{figure}[H]
\centering
\includegraphics[width=0.7\textwidth]{images/system_model_forecasting_demand}
\caption{The system model forecasting schematic representation for electricity demand forecasting.}
\label{fig:modelforecastingdemand}
\end{figure}


\section{Consumption baseline forecasting module}
\label{sec:baselinemodel}
\vspace{0.2 cm}

Describe the consumption baseline forecasting module ...

The system model training schematic representation for consumption baseline forecasting is presented in figure~\ref{fig:modeltrainingbaseline}.

\begin{figure}[H]
\centering
\includegraphics[width=0.7\textwidth]{images/system_model_training_baseline}
\caption{The system model training schematic representation for consumption baseline forecasting.}
\label{fig:modeltrainingbaseline}
\end{figure}

The system model forecasting schematic representation for consumption baseline forecasting is presented in figure~\ref{fig:modelforecastingbaseline}.

\begin{figure}[H]
\centering
\includegraphics[width=0.7\textwidth]{images/system_model_forecasting_baseline}
\caption{The system model forecasting schematic representation for consumption baseline forecasting.}
\label{fig:modelforecastingbaseline}
\end{figure}


\section{Electricity production forecasting module}
\label{sec:productionmodel}
\vspace{0.2 cm}

Describe the electricity production forecasting module ...

The system model training schematic representation for electricity production forecasting is presented in figure~\ref{fig:modeltrainingproduction}.

\begin{figure}[H]
\centering
\includegraphics[width=0.7\textwidth]{images/system_model_training_production}
\caption{The system model training schematic representation for electricity production forecasting.}
\label{fig:modeltrainingproduction}
\end{figure}

The system model forecasting schematic representation for electricity production forecasting is presented in figure~\ref{fig:modelforecastingproduction}.

\begin{figure}[H]
\centering
\includegraphics[width=0.7\textwidth]{images/system_model_forecasting_production}
\caption{The system model forecasting schematic representation for electricity production forecasting.}
\label{fig:modelforecastingproduction}
\end{figure}
