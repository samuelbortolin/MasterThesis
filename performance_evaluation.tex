\chapter{Performance Evaluation}
\label{cha:evaluation}
\vspace{0.4 cm}

Write about the performance evaluation ...

% In this chapter, the proposed system is validated and the results are evaluated.
% Starting by describing the datasets.
% Subsequently, the testing of the system is described.
% Finally, an evaluation of the achieved results is presented for each specific use case.
% After this chapter, it will be clear how the system has been validated and what the results achieved by the proposed system are.

How the chapter is organized ...


\section{MIWEnergia datasets}
\label{sec:datasets}
\vspace{0.2 cm}

Describe the MIWEnergia datasets ...

- Aggregated consumption data: from ... to ..., horly aggregated, ... total entries
- Consumption data of 4 customers: from ... to ..., horly aggregated, ... total entries
- Production data of 8 palnts: from ... to ..., horly aggregated, ... total entries


\section{Evaluation methodology}
\label{sec:methodology}
\vspace{0.2 cm}

General description of the evaluation methodology ...

Describe the metrics (MAPE e MAE) ...

Describe the test of the performance of the models ...

Treat block validation: describe blocked k-fold e hold-out on more periods ...


\section{Electricity demand forecasting}
\label{sec:demandval}
\vspace{0.2 cm}

Analyze the data (descriptive analytics) and the results on the electricity demand forecasting task (predictive analytics) ...

Describe the choice of parameters for models ...


\section{Consumption baseline forecasting}
\label{sec:baselineval}
\vspace{0.2 cm}

Analyze the data (descriptive analytics) and the results on the consumption baseline forecasting task (predictive analytics) ...

Describe the choice of parameters for models ...


\section{Electricity production forecasting}
\label{sec:productionval}
\vspace{0.2 cm}

Analyze the data (descriptive analytics)  and results on the electricity production forecasting task (predictive analytics) ...

Describe the choice of parameters for models ...
