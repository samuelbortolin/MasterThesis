\chapter{Performance Evaluation}
\label{cha:evaluation}
\vspace{0.4 cm}

In this chapter, the proposed system is validated and the performance of the models for the different use cases is evaluated.
The first section presents the datasets provided by MIWEnergia\footnote{ \url{https://www.miwenergia.com/} }.
Subsequently, the adapted evaluation methodology is described.
Finally, the evaluation of the performance of the models for the different use cases is presented.
After this chapter, it will be clear how the system has been validated and what the performance achieved by the proposed system is.


\section{MIWEnergia datasets}
\label{sec:datasets}
\vspace{0.2 cm}

In this section, the MIWEnergia datasets are described.
They provided 3 kinds of datasets: aggregated consumption data from all their customers, consumption data from single customers, and production data from PV plants.

\begin{figure}[H]
\begin{minipage}[b]{8.5cm}
\centering
\includegraphics[width=1\textwidth]{images/demand/data_plot}
\subcaption{}
\label{fig:demanddataplot}
\end{minipage}
\ \hspace{2mm} \
\begin{minipage}[b]{8.5cm}
\centering
\includegraphics[width=1\textwidth]{images/demand/customers_plot}
\subcaption{}
\label{fig:customersplot}
\end{minipage}
\begin{minipage}[b]{17cm}
\centering
\includegraphics[width=0.5\textwidth]{images/demand/mean_data_plot}
\subcaption{}
\label{fig:meandemanddataplot}
\end{minipage}
\caption{The graphical representation of the hourly \subref{fig:demanddataplot} aggregated consumption over customers, \subref{fig:customersplot} number of customers, and \subref{fig:meandemanddataplot} average consumption per customer.}
\end{figure}

The aggregated consumption data from all their customers consists of hourly aggregated consumption data from June 2021 to January 2023 for a total of 14617 entries.
The graphical representation of the hourly aggregated consumption data is reported in figure~\ref{fig:demanddataplot}.
The number of customers is variable, with a maximum of 4253, a minimum of 1591, a mean of 2988, and a standard deviation of 855.
The graphical representation of the hourly number of customers is reported in figure~\ref{fig:customersplot}.
It was thought to normalize the consumption on the basis of the number of customers in order to study the average consumption per user, as illustrated in figure~\ref{fig:meandemanddataplot}, and then multiply by the number of customers.
However, this was not feasible since often this value changes significantly without reflecting on the consumption data, this information was deemed unreliable and not utilized.
Despite this limitation, the average consumption per customer still provided valuable insights into consumption patterns.
Specifically, it suggested the presence of two consumption peaks: one in the summer, likely attributed to air conditioning systems, and a second peak in the winter, likely caused by heating systems.

The consumption data from single customers consists of hourly aggregated consumption data of three customers:
\begin{enumerate}
  \item from June 2021 to August 2022 for a total of 10952 total entries (represented in figure~\ref{fig:dataplotcustomer1});
  \item from September 2021 to May 2022 for a total of 5855 total entries (represented in figure~\ref{fig:dataplotcustomer2});
  \item from September 2021 to August 2022 for a total of 8760 total entries (represented in figure~\ref{fig:dataplotcustomer3}).
\end{enumerate}
The total entries for the three customers are 25567 in the overall period from June 2021 to August 2022.

\begin{figure}[H]
\begin{minipage}[b]{8.5cm}
\centering
\includegraphics[width=1\textwidth]{images/baseline/data_plot_customer1}
\subcaption{First customer.}
\label{fig:dataplotcustomer1}
\end{minipage}
\ \hspace{2mm} \
\begin{minipage}[b]{8.5cm}
\centering
\includegraphics[width=1\textwidth]{images/baseline/data_plot_customer2}
\subcaption{Second customer.}
\label{fig:dataplotcustomer2}
\end{minipage}
\begin{minipage}[b]{17cm}
\centering
\includegraphics[width=0.5\textwidth]{images/baseline/data_plot_customer3}
\subcaption{Third customer.}
\label{fig:dataplotcustomer3}
\end{minipage}
\caption{The graphical representation of the consumption data of the three customers.}
\end{figure}

Only consumption data for three customers were provided, as electricity consumption data of customers is considered personal data as stated in the Directive (EU) 2019/944 of the European Parliament\footnote{ \url{https://eur-lex.europa.eu/legal-content/EN/TXT/?uri=CELEX:32019L0944} }, which establishes common rules for the internal market for electricity.
As such, the data falls under the scope of the General Data Protection Regulation (GDPR)\footnote{ \url{https://gdpr-info.eu/} }, which requires explicit consent for processing personal data.
Therefore, it is likely that the provided data came from MIWEnergia's internal employees who consented to participate in the research.
In fact, the provided data were needed for a technical feasibility study.
However, to identify consumption patterns among a broader population, data from more customers would be required.

By analyzing the data plots, it can be observed that the consumption patterns of the three customers are quite distinct.
For instance:
\begin{itemize}
  \item The first customer's consumption is consistently below 1.5 kWh throughout the year except for the winter season;
  \item The second customer shows almost the same consumption pattern;
  \item The third customer has a dense series with some gaps with very low consumption in parts of the year, probably due to time periods away from home.
\end{itemize}

The production data from 8 PV plants consists of hourly aggregated production data:
\begin{enumerate}
  \item from January 2022 to October 2022 for a total of 7296 total entries and it has a nominal power of 149.75 kW;
  \item from February 2022 to October 2022 for a total of 6552 total entries and it has a nominal power of 237.6 kW;
  \item from February 2022 to October 2022 for a total of 6552 total entries and it has a nominal power of 158.4 kW;
  \item from June 2022 to October 2022 for a total of 3576 total entries and it has a nominal power of 1240 kW;
  \item from September 2022 to October 2022 for a total of 1465 total entries and it has a nominal power of 126.2 kW;
  \item from September 2022 to October 2022 for a total of 1465 total entries and it has a nominal power of 113 kW;
  \item from September 2022 to October 2022 for a total of 1465 total entries and it has a nominal power of 45 kW;
  \item from September 2022 to October 2022 for a total of 1465 total entries and it has a nominal power of 100 kW;
\end{enumerate}
The total entries for the 8 PV plants are 29836 total entries in the overall period from January 2022 to October 2022.
The graphical representation of the hourly aggregated total and mean percentage production data are reported respectively in figure~\ref{fig:productiondataplot} and figure~\ref{fig:productiondataplotpercentage}.

\begin{figure}[H]
\begin{minipage}[b]{8.5cm}
\centering
\includegraphics[width=1\textwidth]{images/production/data_plot}
\subcaption{}
\label{fig:productiondataplot}
\end{minipage}
\ \hspace{2mm} \
\begin{minipage}[b]{8.5cm}
\centering
\includegraphics[width=1\textwidth]{images/production/data_plot_percentage}
\subcaption{}
\label{fig:productiondataplotpercentage}
\end{minipage}
\caption{The graphical representation of the hourly aggregated \subref{fig:umlsingleplant} total and \subref{fig:umlsingleplant} mean percentage production data.}
\end{figure}

The same weather data from the same provider and the same weather station were used for all the tasks.
While it is reasonable to assume that some weather conditions, such as clouds, may vary and affect consumption and production differently, this approach was taken since the customers and the PV plants were in the same area, with a maximum distance of around 100 km between them, and also because the value of the weather parameters is an average of the values recorded in the past hour.
Therefore, using the same weather data is deemed appropriate for the analysis at hand, given that it reflects the weather conditions in the area over a broad time frame.


\section{Evaluation methodology}
\label{sec:methodology}
\vspace{0.2 cm}

The evaluation methodology was based on two relevant error metrics: Mean Absolute Percentage Error (MAPE) and Mean Absolute Error (MAE).
These are standard and widely used metrics in time series forecasting for different use cases, as reported in many articles and books such as \cite{armstrong2001principles, DEGOOIJER2006443, HYNDMAN2006679}.
The MAPE is defined as $\text{MAPE}(y, \hat{y}) = \frac{100\%}{N} \sum_{i=0}^{N - 1} \frac{|y_i - \hat{y}_i|}{|y_i|}$.
It is the most relevant error metric for all the tasks since it percentage-based error metric that takes into account the magnitude of the errors relative to the actual values.
The MAE is defined as $\text{MAE}(y, \hat{y}) = \frac{ \sum_{i=0}^{N - 1} |y_i - \hat{y}_i| }{N}$.
It is the most suitable error metric in consumption baseline forecasting where there is a high variability from very low to high values, and electricity production forecasting where there is a significant number of zeros when the sun is absent.

For assessing the model performance when dealing with time series data, the traditional cross-validation techniques are not suitable as they assume that the data points are independent and identically distributed (i.i.d), which is not the case in time series data, as also reported in the chapter~\ref{cha:soa} by the following articles \cite{BERGMEIR2012192, Cerqueira2020}.
In fact, in time series data the order of the observations matters, and there is a temporal dependencies between the observations.
Therefore, a more appropriate technique for evaluating time series models is time series cross-validation.
The basic idea of time series cross-validation is to split the data into multiple training and test sets, where the training set only includes data from the past and the test set includes data from the future.
This simulates the real-world scenario where we want to make predictions about the future based on past data.
In figure~\ref{fig:crossvalidation}, the schematic representation of how the cross-validation was performed is reported, 12 splits were used with a test size depending on the specific use case.

\begin{figure}[H]
\centering
\includegraphics[width=0.8\textwidth]{images/cross_validation}
\caption{The schematic representation of the blocked k-fold validation adopted.}
\label{fig:crossvalidation}
\end{figure}

Another technique, reported in \cite{Cerqueira2020} is the repeated Holdout Out‐of‐sample tested in multiple testing periods with a Monte Carlo simulation using 70\% of the total observations of the time series in each test.
For each period, a random point is picked from the time series.
The previous window comprising 60\% of the time series is used for training and the following window of 10\% of the time series is used for testing.
In the paper was stated that the approach provided the most accurate estimates when the time series are non-stationary.
However, for having the same evaluation mechanisms on all the models and simulating the fact that the model starts with limited data and then the training amount increase over time to obtain better performance, block validation was used using the TimeSeriesSplit provided by the scikit-learn library.
Moreover, the considered data are without a strong trend as reported in the dedicated sections of the specific use cases where the data and the model forecasts are analyzed.

In addition to the cross-validation results also the results on the last split using the rest as training are reported.
This was done because it provides insight on which could be the performance of the models in forecasting the near future data with the currently available training data.


\section{Electricity demand forecasting}
\label{sec:demandval}
\vspace{0.2 cm}

Analyze the data (descriptive analytics) and the results of the models for the electricity demand forecasting task (predictive analytics) ...

Data stats and correlation with weather data ...

Basic data is enhanced with the air temperature, the apparent temperature, and the relative humidity.

Describe the choice of parameters for models ... (list with motivations)

12 splits were used with a test size of 720 each time.
Basically starting with 5977 entries and predicting every time 720 entries up to reach the last prediction instant where the model has a training size of 13.897.

Summary table with results ...
% TODO where results are the one given by cross validation mean ± std
% TODO check the consistency of model results in the last period and using cross-validation --> do a table with the differences (+ plot for internal analysis)


\section{Electricity production forecasting}
\label{sec:productionval}
\vspace{0.2 cm}

Analyze the data (descriptive analytics) and the results of the models for the electricity production forecasting task (predictive analytics) ...

As described in the data preprocessing in chapter~\ref{cha:implementation}, single PV plant production data are aggregated to obtain the aggregated production data over the PV plants.
The target of the predictions is the mean percentage of production, which is calculated as the division of the total produced energy by the total power of the PV plants.
This allows to have a bounded value from 0 to 100 from which it is possible to obtain the total produced energy simply by multiplying it by the total power of the PV plants.
(This was also done since PV plants are added over time and this was unpredictable, this results in predicting the percentage of production of the PV plants.)

Data stats and correlation with weather data ...

Basic data is enhanced with the air temperature, the apparent temperature, the relative humidity, the wind speed, the wind direction, the pressure altimeter, the visibility, the sky coverage, the diffuse horizontal irradiance, the direct normal irradiance, the global horizontal irradiance, the solar radiation, the UV index, the solar elevation angle, and the solar azimuth angle.
Only the hourly granularity is considered since PV plants are highly correlated with weather data and the aggregation over the day loses this correlation.
If we were forecasting the production of each PV plant individually, the weather data at the specific locations of the plants would result in a higher correlation with the production data and likely allow the models to produce more accurate forecasts.
However, since we are aggregating the production from multiple PV plants and looking for the overall percentage of production in the hour, using the weather data from a location located in the middle of the PV plants is also acceptable.
In figure~\ref{fig:pvplantsmap}, the locations of the PV plants and of the weather station are reported.

\begin{figure}[H]
\centering
\includegraphics[width=0.8\textwidth]{images/production/pv_plants_map}
\caption{The map of the city of Murcia with indicated the location of the PV plants with a blue icon and of the revelation station with a lightblue icon. The distances between each PV plant and the weather station are reported in lightblue boxes near the PV plant locations.}
\label{fig:pvplantsmap}
\end{figure}

Describe the choice of parameters for models ... (list with motivations)

Summary table with results ...
% TODO where results are the one given by cross validation mean ± std
% TODO check the consistency of model results in the last period and using cross-validation --> do a table with the differences (+ plot for internal analysis)


\section{Consumption baseline forecasting} 
\label{sec:baselineval}
\vspace{0.2 cm}

Analyze the data (descriptive analytics) and the results of the models for the consumption baseline forecasting task (predictive analytics) ...

Data stats and correlation with weather data ...

As can be noticed from the data, there is high variability in consumption and low auto-correlation, this suggests how it is difficult to produce highly accurate results on a single customer level.
With just the time series of a few users, it is very difficult to learn a well-performing model, having more users it could be possible to learn certain generic trends or standard behaviors.

Basic data is enhanced with the air temperature, the apparent temperature, and the relative humidity.

Describe the choice of parameters for models ... (list with motivations)

Summary table with results ...
% TODO where results are the one given by cross validation mean ± std
% TODO check the consistency of model results in the last period and using cross-validation --> do a table with the differences (+ plot for internal analysis)
